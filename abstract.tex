%!TEX root = main.tex

\begin{abstract}
Log-structured merge-trees (LSM trees) are increasingly used as the 
    storage engines behind several data systems, many of 
    which are deployed in the cloud. 
% Similar to other database architectures, LSM trees rely on tuning
%     to optimize their performance taking into account the \emph{expected} workload (e.g., reads vs. 
%     writes and point vs. range queries). 
Similar to other database architectures, LSM trees take into account
    information about the \emph{expected} workloads (e.g., reads vs. writes and
    point vs. range queries) and optimize their performances by changing tunings.
%and the \emph{available resources} (e.g., amount of main memory, storage performance).
Operating in the cloud, however, comes with a degree of 
    \emph{uncertainty} due to multi-tenancy and the fast-evolving nature
    of modern applications.
Databases with static tunings discount the variability of such hybrid workloads
    and hence provide an inconsistent and overall suboptimal performance.
% Hybrid workloads can be dynamic having significant variability from the expectation, leading
%     to suboptimal tuning with subpar performance.

%
%a specific tuning suboptimal. At the
%same time, the available resources may fluctuate especially due to multi-tenancy. In both cases,
%we observe an \emph{uncertainty} with respect to the assumption when tuning the underlying storage
%engine, which often leads to suboptimal performance.  

To address this problem, we introduce {\Endure} -- a new paradigm
    for tuning LSM Trees in the presence of workload uncertainty.
% and we apply it to LSM trees. 
Specifically, we focus on the impact of the choice of compaction policies, size-ratio,
    and memory allocation on the overall query performance.
% Specifically, we focus on the impact of the compaction policy, the size ratio, and 
% the memory allocation on the ingestion and query performance of LSM trees. 
{\Endure} considers a robust formulation of the throughput maximization problem,
    and recommends a tuning
    that maximizes the worst-case throughput over the \emph{neighborhood}
    of an expected workload.
% We frame the LSM tuning
% process, as a robust optimization problem that finds a 
% tuning that is the best for the worst-case workload among those that are in the
% \emph{neighborhood} of the expected workload. 
Additionally, an uncertainty tuning parameter controls the size of this 
    neighborhood, thereby allowing the output tunings to be conservative or
    optimistic.
% The size of this neighborhood is given as an input 
% to the robust tuning process, allowing the robust tuning to 
% be conservative or optimistic accordingly. 
We benchmark {\Endure} on a state-of-the-art LSM-based storage engine, RocksDB,
    and show that its tunings comprehensively outperform tunings from classical
    strategies.
Drawing upon the results of our extensive analytical and empirical evaluation, 
    we recommend the use of {\Endure} for optimizing the performance of 
    LSM tree-based storage engines.
% and we show both analytically and empirically that 
% robust tuning outperforms
% classical tuning even when there are small discrepancies 
% between the expected and the observed workloads. 

\end{abstract}
