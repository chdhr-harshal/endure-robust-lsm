\section{Problem Definitions}
In this section, we provide the formal problem definitions on how to choose the \emph{design parameters} of 
an LSM tree. Before proceeding, we give a brief introduction to our notation.

\subsection{Notation}\label{sec:notation}
As we discussed above, LSM trees have two types of parameters: the \emph{design parameters} and the \emph{system parameters}.  One can think of the design parameters as those that someone who aims to optimize the performance of
an LSM tree can tune, while the system parameters are given and therefore untunable.

\Paragraph{Design Parameters}
The design parameters we consider in this paper (in accordance with the related
    work~\cite{Dayan2017,Dayan2018a,Luo2020b}) are the  size-ratio ({\sizeratio}),
    the memory allocated to the Bloom filters ({\mfilt}), the memory allocated to
    the write buffer ({\mbuf}) and the policy ({\policy}) as shown in Table
    \ref{tab:model-design-params}.
Recall that the policy refers to either leveling or tiering, as discussed in the previous section.


\begin{table}[h]\centering%\small
    %\vspace{-0.1in}
\renewcommand{\arraystretch}{1.1}
\begin{tabular}{cl}
    \toprule
    Term & Definition \\
    \midrule
        $\mfilt$ &  Memory allocated for Bloom filters  \\ 
        $\mbuf$ &  Memory allocated for the write buffer \\ 
        $\sizeratio$ &  Size ratio between consecutive levels   \\ 
        \policy & Compaction policy (\emph{tiering}/\emph{leveling}) \\ 
    \bottomrule
\end{tabular}
\caption{\emph{Design} parameters of an LSM tree.}
	\label{tab:model-design-params}	
	%\vspace{-0.35in}
\end{table}
 
\Paragraph{System Parameters} A complicated data structure like LSM trees also has other various \emph{system parameters} and other non-tunable as shown in Table \ref{tab:model-system-params} (e.g., total memory ({\mtot}), size of data entries $E$, page size $B$, data size $N$). 

\begin{table}[h]\centering%\small
    %\vspace{-0.1in}
\renewcommand{\arraystretch}{1.1}
\resizebox{\columnwidth}{!}{
\begin{tabular}{cl}
    \toprule
    Term & Definition \\
    \midrule
    $\mtot$ & Total memory (Bloom filters+write buffer)    ($\mtot=\mbuf+\mfilt$)\\
        $E$ &  Size of a key-value entry  \\ 
        $B$ &  Number of entries that fit in a page   \\ 
        $N$ & Total number of entries    \\ 	
    \bottomrule
\end{tabular}
}
\caption{\emph{System} and untunable parameters of an LSM tree.}
	\label{tab:model-system-params}	
%	\vspace{-0.05in}
\end{table}
    
\Paragraph{LSM Tree Configuration} In terms of notation we use $\configuration$ to denote the 
LSM tree tuning configuration which essentially describes the values of the
tunable parameters together  $\configuration := (\sizeratio, \mfilt, \policy)$. Note that we only
use the memory for Bloom filters $\mfilt$ and not 
$\mbuf$, because the latter can be derived using the 
former and the 
total available memory: $\mbuf=\mtot-\mfilt$.

\Paragraph{Workload}  The choice of the parameters in $\configuration$ 
depends on the input (expected) workload, i.e., the fraction of
empty lookups
    ({\emptylookup}), non-empty lookups ({\nonemptylookup}), range lookups
    ({\range}), and write ({\update}) queries, as shown in Table \ref{tab:workload-params}.
A workload can therefore be expressed as a vector 
    ${\workload = (\emptylookup, \nonemptylookup, \range, \update)^\intercal \geq 0}$
    describing the proportions of the different kinds of queries.
Clearly, $\emptylookup+\nonemptylookup+\range+\update = 1$ or alternatively:
$\workload^\intercal \columnvec = 1$ where {\columnvec} denotes a column vector of
    ones. %of the same dimension as {\workload}.

\begin{table}[h]\centering%\small
    %\vspace{-0.1in}
\renewcommand{\arraystretch}{1.1}
\begin{tabular}{cl}
    \toprule
    Term & Definition \\
    \midrule
        $z_0$ & Percentage of zero-result point lookups\\ %in the workload \\ 
        $z_1$ & Percentage of non-zero-result point lookups\\ %in the workload  \\ 
        $q$ & Percentage of range queries\\ %in the workload  \\ 
        $w$ & Percentage of updates\\ %in the workload  \\ 
    \bottomrule
\end{tabular}
\caption{Parameters describing the \emph{workload}.}
    \label{tab:workload-params}	
    %\vspace{-0.35in}
\end{table}

\noindent Each type of query (non-empty lookups, empty lookups, range lookups and writes) has a different cost, denoted as
$Z_0(\configuration)$, $Z_1(\configuration)$, $Q(\configuration)$, $W(\configuration)$, as there is a dependency between
the cost of each type of query and the design ${\configuration}$.
For easiness of notation, we use 
    $\costvec(\configuration) = \left(Z_0(\configuration), Z_1(\configuration), Q(\configuration), W(\configuration)\right)^\intercal$ 
    to denote the vector of the costs  of executing
    different types of queries.
Thus, given a specific configuration ({\configuration}) and a workload ({\workload}),
    the expected cost for the workload can be computed as:
{%
%\small
\begin{equation}
    \label{eq:thecost}
    \cost(\workload, \configuration) = \workload^\intercal
    \costvec(\configuration)=\nonemptylookup \cdot
    Z_0(\configuration)+\emptylookup \cdot Z_1(\configuration) + \range\cdot Q(\configuration) + \update \cdot W(\configuration).
\end{equation}
}%

\subsection{The Nominal Tuning Problem}
Traditionally, the designers have focused on finding the 
configuration ${\configuration}^\ast$ that minimizes the total cost %$\cost(\workload, \configuration^\ast;\varphi)$, 
$\cost(\workload, \configuration^\ast)$,
 for a given fixed workload $\workload$.  We call this problem the
 {\nominal} problem, defined as follows:
 
 \begin{problem}[{\nominal}]\label{problem:nominal}
 Given fixed $\workload$  find the tuning configuration of the LSM tree $\configuration_N$ such that
\begin{equation}
\label{eq:nominal_problem}
    \configuration_N = \argmin_{\configuration} \cost(\workload, \configuration).
\end{equation}
\end{problem}

\noindent The nominal tuning problem described above 
captures the classical tuning paradigm. It uses a
cost-model to find a system configuration that 
minimizes the cost given a specific workload and 
system environment. Specifically, prior tuning
approaches for LSM trees solve the nominal tuning
problem when proposing optimal memory allocation,
and merging policies \cite{Dayan2017,Dayan2018a,Luo2020a}.



\subsection{The Robust Tuning Problem}
\label{subsec:robust-tuning}

In this work, we attempt to compute high-performance configurations that minimize
    expected cost of operation, as expressed in Equation~\eqref{eq:thecost}, 
    in the presence of uncertainty with respect to the expected workload.

%\Paragraph{Uncertain Workload:}
In the {\nominal} problem, the designers assume perfect information about the 
    workload for which to tune the system. For example, they may assume that the
    input vector $\workload$ represents the workload for which they have to
    optimize. While in practice, $\workload$ is simply an estimate of what the
    workload will look like.
Hence, the configuration obtained by solving Problem~\ref{problem:nominal} may
    result in high variability in the system performance; which will inevitably
    depend on the actual observed workload upon the deployment of the system. 
    
We capture this uncertainty by reformulating Problem~\ref{problem:nominal} to
    take into account the variability that can be observed in the input workload. 
Given expected workload $\workload$, we introduce the notion of the
    \emph{uncertainty region} of $\workload$, which we denote by
    $\mathcal{U}_\workload$.

We can define the robust version of Problem~\ref{problem:nominal}, under the 
assumption that there is uncertainty in the input workload as follows:

\begin{problem}[{\robustw}]\label{problem:robustw} 
Given $\workload$ and uncertainty region $\mathcal{U}_\workload$ 
find the tuning configuration of the LSM tree $\configuration_R$ such that
\begin{eqnarray}
\label{eq:robust_workload_problem}
\configuration_R &=& \argmin_{\configuration} \cost(\obsworkload,
    \configuration) \nonumber\\
    \textrm{s.t.,}&& \obsworkload \in \mathcal{U}_{\workload}.
\end{eqnarray}
\end{problem}

Note that the above problem definition intuitively states the following: it recognizes that the input workload $\workload$ won't
be observed exactly, and it assumes that any workload in $\mathcal{U}_\workload$ is possible.  Then, it searches for the
configuration $\configuration_\workload$ that is best for the \emph{worst-case} scenario among all those in $\mathcal{U}_\workload$. 

The challenge in solving {\robustw}  is that one needs to explore all the workloads in the uncertainty region
in order to solve the problem.  In the next section, we show that this is not necessary.  In fact, by appropriately rewriting 
the problem definition we show that we can solve Problem~\ref{problem:robustw} in polynomial time.


%
%$\mathcal{U}_{\workload}$ represents the \emph{uncertainty region} around
%    the expected workload {\workload}; a set of  workloads similar to but
%    not exactly the same as {\workload}. 
%In Section~\ref{sec:robust_lsm_tuning}, we explain in detail the
%    parameterization of the uncertainty region 
%    based on the number of sample workloads used during the computation of the 
%    expected workload and the confidence of the designer on how relevant the 
%    observed samples are to the future workloads.
